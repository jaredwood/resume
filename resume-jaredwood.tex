%%%%%%%%%%%%%%%%%%%%%%%%%%%%%%%%%%%%%%%%%%%%%%%%%%%%%%%%%%%%%%%%%%%%%%%%
%%%%%%%%%%%%%%%%%%%%%% Simple LaTeX CV Template %%%%%%%%%%%%%%%%%%%%%%%%
%%%%%%%%%%%%%%%%%%%%%%%%%%%%%%%%%%%%%%%%%%%%%%%%%%%%%%%%%%%%%%%%%%%%%%%%

%%%%%%%%%%%%%%%%%%%%%%%%%%%%%%%%%%%%%%%%%%%%%%%%%%%%%%%%%%%%%%%%%%%%%%%%
%% NOTE: If you find that it says                                     %%
%%                                                                    %%
%%                           1 of ??                                  %%
%%                                                                    %%
%% at the bottom of your first page, this means that the AUX file     %%
%% was not available when you ran LaTeX on this source. Simply RERUN  %%
%% LaTeX to get the ``??'' replaced with the number of the last page  %%
%% of the document. The AUX file will be generated on the first run   %%
%% of LaTeX and used on the second run to fill in all of the          %%
%% references.                                                        %%
%%%%%%%%%%%%%%%%%%%%%%%%%%%%%%%%%%%%%%%%%%%%%%%%%%%%%%%%%%%%%%%%%%%%%%%%

%%%%%%%%%%%%%%%%%%%%%%%%%%%% Document Setup %%%%%%%%%%%%%%%%%%%%%%%%%%%%

% Don't like 10pt? Try 11pt or 12pt
\documentclass[10pt]{article}

% This is a helpful package that puts math inside length specifications
\usepackage{calc}


% Simpler bibsection for CV sections
% (thanks to natbib for inspiration)
\makeatletter
\newlength{\bibhang}
\setlength{\bibhang}{1em}
\newlength{\bibsep}
 {\@listi \global\bibsep\itemsep \global\advance\bibsep by\parsep}
\newenvironment{bibsection}%
        {\vspace{-\baselineskip}\begin{list}{}{%
       \setlength{\leftmargin}{\bibhang}%
       \setlength{\itemindent}{-\leftmargin}%
       \setlength{\itemsep}{\bibsep}%
       \setlength{\parsep}{\z@}%
        \setlength{\partopsep}{0pt}%
        \setlength{\topsep}{0pt}}}
        {\end{list}\vspace{-.6\baselineskip}}
\makeatother

% Layout: Puts the section titles on left side of page
\reversemarginpar

%
%         PAPER SIZE, PAGE NUMBER, AND DOCUMENT LAYOUT NOTES:
%
% The next \usepackage line changes the layout for CV style section
% headings as marginal notes. It also sets up the paper size as either
% letter or A4. By default, letter was used. If A4 paper is desired,
% comment out the letterpaper lines and uncomment the a4paper lines.
%
% As you can see, the margin widths and section title widths can be
% easily adjusted.
%
% ALSO: Notice that the includefoot option can be commented OUT in order
% to put the PAGE NUMBER *IN* the bottom margin. This will make the
% effective text area larger.
%
% IF YOU WISH TO REMOVE THE ``of LASTPAGE'' next to each page number,
% see the note about the +LP and -LP lines below. Comment out the +LP
% and uncomment the -LP.
%
% IF YOU WISH TO REMOVE PAGE NUMBERS, be sure that the includefoot line
% is uncommented and ALSO uncomment the \pagestyle{empty} a few lines
% below.
%

%% Use these lines for letter-sized paper
\usepackage[paper=letterpaper,
            %includefoot, % Uncomment to put page number above margin
            marginparwidth=1.2in,     % Length of section titles
            marginparsep=.05in,       % Space between titles and text
            margin=1in,               % 1 inch margins
            includemp]{geometry}

%% Use these lines for A4-sized paper
%\usepackage[paper=a4paper,
%            %includefoot, % Uncomment to put page number above margin
%            marginparwidth=30.5mm,    % Length of section titles
%            marginparsep=1.5mm,       % Space between titles and text
%            margin=25mm,              % 25mm margins
%            includemp]{geometry}

%% More layout: Get rid of indenting throughout entire document
\setlength{\parindent}{0in}

%% This gives us fun enumeration environments. compactitem will be nice.
\usepackage{paralist}

%% Reference the last page in the page number
%
% NOTE: comment the +LP line and uncomment the -LP line to have page
%       numbers without the ``of ##'' last page reference)
%
% NOTE: uncomment the \pagestyle{empty} line to get rid of all page
%       numbers (make sure includefoot is commented out above)
%
\usepackage{fancyhdr,lastpage}
\pagestyle{fancy}
%\pagestyle{empty}      % Uncomment this to get rid of page numbers
\fancyhf{}\renewcommand{\headrulewidth}{0pt}
\fancyfootoffset{\marginparsep+\marginparwidth}
\newlength{\footpageshift}
\setlength{\footpageshift}
          {0.5\textwidth+0.5\marginparsep+0.5\marginparwidth-2in}
\lfoot{\hspace{\footpageshift}%
       \parbox{4in}{\, \hfill %
                    \arabic{page} of \protect\pageref*{LastPage} % +LP
%                    \arabic{page}                               % -LP
                    \hfill \,}}

% Finally, give us PDF bookmarks
\usepackage{color,hyperref}
%\definecolor{darkblue}{rgb}{0.0,0.0,0.3}
\definecolor{black}{rgb}{0,0,0}
\hypersetup{colorlinks,breaklinks,
            linkcolor=black,urlcolor=black,
            anchorcolor=black,citecolor=black}


%%%%%%%%%%%%%%%%%%%%%%%% End Document Setup %%%%%%%%%%%%%%%%%%%%%%%%%%%%


%%%%%%%%%%%%%%%%%%%%%%%%%%% Helper Commands %%%%%%%%%%%%%%%%%%%%%%%%%%%%

% The title (name) with a horizontal rule under it
% (optional argument typesets an object right-justified across from name
%  as well)
%
% Usage: \makeheading{name}
%        OR
%        \makeheading[right_object]{name}
%
% Place at top of document. It should be the first thing.
% If ``right_object'' is provided in the square-braced optional
% argument, it will be right justified on the same line as ``name'' at
% the top of the CV. For example:
%
%       \makeheading[\emph{Curriculum vitae}]{Your Name}
%
% will put an emphasized ``Curriculum vitae'' at the top of the document
% as a title. Likewise, a picture could be included:
%
%   \makeheading[\includegraphics[height=1.5in]{my_picutre}]{Your Name}
%
% the picture will be flush right across from the name.
\newcommand{\makeheading}[2][]%
        {\hspace*{-\marginparsep minus \marginparwidth}%
         \begin{minipage}[t]{\textwidth+\marginparwidth+\marginparsep}%
             {\large \bfseries #2 \hfill #1}\\[-0.15\baselineskip]%
                 \rule{\columnwidth}{1pt}%
         \end{minipage}}

% The section headings
%
% Usage: \section{section name}
%
% Follow this section IMMEDIATELY with the first line of the section
% text. Do not put whitespace in between. That is, do this:
%
%       \section{My Information}
%       Here is my information.
%
% and NOT this:
%
%       \section{My Information}
%
%       Here is my information.
%
% Otherwise the top of the section header will not line up with the top
% of the section. Of course, using a single comment character (%) on
% empty lines allows for the function of the first example with the
% readability of the second example.
\renewcommand{\section}[2]%
        {\pagebreak[3]\vspace{1.3\baselineskip}%
         \phantomsection\addcontentsline{toc}{section}{#1}%
         \hspace{0in}%
         \marginpar{
         \raggedright \scshape #1}#2}

% An itemize-style list with lots of space between items
\newenvironment{outerlist}[1][\enskip\textbullet]%
        {\begin{itemize}[#1]}{\end{itemize}%
         \vspace{-.6\baselineskip}}

% An environment IDENTICAL to outerlist that has better pre-list spacing
% when used as the first thing in a \section
\newenvironment{lonelist}[1][\enskip\textbullet]%
        {\vspace{-\baselineskip}\begin{list}{#1}{%
        \setlength{\partopsep}{0pt}%
        \setlength{\topsep}{0pt}}}
        {\end{list}\vspace{-.6\baselineskip}}

% An itemize-style list with little space between items
\newenvironment{innerlist}[1][\enskip\textbullet]%
        {\begin{compactitem}[#1]}{\end{compactitem}}

% An environment IDENTICAL to innerlist that has better pre-list spacing
% when used as the first thing in a \section
\newenvironment{loneinnerlist}[1][\enskip\textbullet]%
        {\vspace{-\baselineskip}\begin{compactitem}[#1]}
        {\end{compactitem}\vspace{-.6\baselineskip}}

% To add some paragraph space between lines.
% This also tells LaTeX to preferably break a page on one of these gaps
% if there is a needed pagebreak nearby.
\newcommand{\blankline}{\quad\pagebreak[3]}
\newcommand{\halfblankline}{\quad\vspace{-0.5\baselineskip}\pagebreak[3]}

% Uses hyperref to link DOI
\newcommand\doilink[1]{\href{http://dx.doi.org/#1}{#1}}
\newcommand\doi[1]{doi:\doilink{#1}}

% For \url{SOME_URL}, links SOME_URL to the url SOME_URL
\providecommand*\url[1]{\href{#1}{#1}}
% Same as above, but pretty-prints SOME_URL in teletype fixed-width font
\renewcommand*\url[1]{\href{#1}{\texttt{#1}}}

% For \email{ADDRESS}, links ADDRESS to the url mailto:ADDRESS
\providecommand*\email[1]{\href{mailto:#1}{#1}}
% Same as above, but pretty-prints ADDRESS in teletype fixed-width font
%\renewcommand*\email[1]{\href{mailto:#1}{\texttt{#1}}}

%\providecommand\BibTeX{{\rm B\kern-.05em{\sc i\kern-.025em b}\kern-.08em
%    T\kern-.1667em\lower.7ex\hbox{E}\kern-.125emX}}
%\providecommand\BibTeX{{\rm B\kern-.05em{\sc i\kern-.025em b}\kern-.08em
%    \TeX}}
\providecommand\BibTeX{{B\kern-.05em{\sc i\kern-.025em b}\kern-.08em
    \TeX}}
\providecommand\Matlab{\textsc{Matlab}}

%%%%%%%%%%%%%%%%%%%%%%%% End Helper Commands %%%%%%%%%%%%%%%%%%%%%%%%%%%

%%%%%%%%%%%%%%%%%%%%%%%%% Begin CV Document %%%%%%%%%%%%%%%%%%%%%%%%%%%%

\begin{document}
%\makeheading[\emph{Resume}]{Jared G.~Wood}
\makeheading[]{Jared G.~Wood}

\section{Contact Information}
%
% NOTE: Mind where the & separators and \\ breaks are in the following
%       table.
%
% ALSO: \rcollength is the width of the right column of the table
%       (adjust it to your liking; default is 1.85in).
%
\newlength{\rcollength}\setlength{\rcollength}{2.5in}%
%
\begin{tabular}[t]{@{}p{\textwidth-\rcollength}p{\rcollength}}
%\href{http://www.me.berkeley.edu/}%
%     {Department of Mechanical Engineering} & \\
%\href{http://www.berkeley.edu/}{Univerisity of California, Berkeley} \\
%6216 Antioch St Apt C	& 510-610-7029 \\
%Oakland, CA  94611 USA	& \email{jared.jwood@gmail.com}\\
510-610-7029 \\
\email{jared.jwood@gmail.com} \\
%Berkeley, CA  94720 USA				& \textit{WWW:}
%\href{http://vehicle.berkeley.edu/~jwood}{vehicle.berkeley.edu/{\raise.17ex\hbox{$\scriptstyle\sim$}}jwood}\\
\end{tabular}

%\section{Objective}
%% This objective is aimed at R&D staff.
An experienced mobile robotics researcher, algorithm designer, software developer, and team leader seeking to join a research and development team. Strong aptitude for tackling complex problems.


%\section{Availability}%
%\begin{loneinnerlist}
%    \item Available immediately.
%    \item Geographic location is flexible.
%\end{loneinnerlist}

%\section{Security Clearance}
%%
%Department of Defense Top Secret SCI with polygraph (expired: 2002)

%\section{Citizenship}
%%
%USA

\section{Background}
Experience designing, developing, implementing, and testing algorithms for navigation-related artificial intelligence and learning.


%\section{Research Interests}
\section{Fields of Interest}
%Autonomous systems/robotics; navigation and interaction with uncertain environment; path planning, control, estimation/localization, perception, sensor fusion.

%Artificial intelligence, probabilistic modeling, machine learning, big data.

(Deep) supervised and reinforcement learning; autonomous robotics; perception and navigation; leadership/management.


\section{Education}
%
\href{http://www.berkeley.edu/}{\textbf{University of California$-$Berkeley}}\\
Berkeley, California USA
\begin{outerlist}
\item[] Ph.D.
		Engineering \hfill \textbf{Dec 2011}
        \begin{innerlist}
        \item Fields: Artificial Intelligence, Machine Learning, Control Systems.
        \item Research: Probabilistic vision-based target tracking from autonomous aircraft.
        \item Adviser:
              \href{http://www.me.berkeley.edu/faculty/hedrick}
                   {Professor J. Karl~Hedrick}.
        \item Associations:
        \begin{innerlist}
        	\item[$\diamond$] \href{http://vehicle.berkeley.edu}{Vehicle Dynamics Lab}
        	\item[$\diamond$] \href{http://c3uv.berkeley.edu}{Center for Collaborative Control of Unmanned Vehicles}
        \end{innerlist}
        \end{innerlist}
\item[] M.S.
        Engineering \hfill \textbf{May 2008}
        \begin{innerlist}
        \item Fields: Control Systems, Sensor Networks.
        \item Research: Wireless distributed sensor networks.
        \item Adviser:
              \href{http://www.me.berkeley.edu/faculty/auslander}
                   {Professor David M.~Auslander}\\
        \end{innerlist}
\end{outerlist}

\href{http://www.utah.edu/}{\textbf{University of Utah}}\\
Salt Lake City, Utah USA
\begin{outerlist}
\item[] B.S. \href{http://www.mech.utah.edu/}{Mechanical Engineering}
		\hfill \textbf{May 2006}
        \begin{innerlist}
        \item Minor: Mathematics.
        \item \emph{Cum Laude}, with Honors in Engineering.
        \end{innerlist}
\end{outerlist}


\section{Awards}
%
\href{http://www.berkeley.edu}{\textbf{University of California$-$Berkeley}}
\begin{innerlist}
\item Block Grant Award, Department of Mechanical Engineering \hfill \textbf{2011}
\end{innerlist}

\halfblankline

\href{http://www.utah.edu}{\textbf{The University of Utah}}
\begin{innerlist}
\item Gerald J. Gagner Scholarship, College of Engineering \hfill \textbf{2005}
\item John Murray Endowed Scholarship, College of Engineering \hfill \textbf{2005}
\item Full Tuition Waiver, Department of Mechanical Engineering \hfill \textbf{2004}
\end{innerlist}


%\section{Software Experience}
%\input{experience-software}

\section{Work Experience}
%
\textbf{Autonomous Mobile Robotics Startup}\\
Oakland, California USA
\begin{outerlist}
\item[] \textit{Probabilistic Roboticist}%
        \hfill \textbf{Feb 2014 to current}
\begin{innerlist}
\item In charge of developing autonomous navigation algorithms.
\item Algorithms include perception for terrain/obstacle detection and avoidance, localization and mapping. Implementing in C++.
\end{innerlist}
\end{outerlist}

\halfblankline

\textbf{Automa Aurora}\\
Berkeley, California USA
\begin{outerlist}
\item[] \textit{Algorithm/Software Developer}%
        \hfill \textbf{Jun 2012 to Oct 2013}
\begin{innerlist}
\item Two-man startup; designed and built distributed optimization system for vehicle routing.
\item Distributed system was built and deployed with customers. Implemented in Java.
\end{innerlist}
\end{outerlist}

\halfblankline

\href{http://www.utrc.utc.com/}{\textbf{United Technologies Research Center at Berkeley}}\\
Berkeley, California USA
\begin{outerlist}
\item[] \textit{Research and Development}%
        \hfill \textbf{Sep 2011 to Apr 2012}
\begin{innerlist}
\item Developed algorithms for vision-based target tracking from autonomous helicopter.
\item Algorithms included particle filter target tracking and likelihood functions for vision-based perception. Implemented in C++.
\end{innerlist}
\end{outerlist}

\halfblankline

%
\href{http://c3uv.berkeley.edu/}{\textbf{Center for Collaborative Control of Unmanned Vehicles}}\\
\textbf{University of California$-$Berkeley}\\
Berkeley, California USA
\begin{outerlist}
\item[] \textit{Researcher}%
        \hfill \textbf{Aug 2007 to Sep 2011}
\begin{innerlist}
\item Several projects for autonomous aircraft artificial intelligence and navigation.
\item Developed algorithms for tracking dynamic environments using vision-based perception, developed information-based search methods for path planning. Implemented in C++. Tested in several experiments on-board real aircraft.
\end{innerlist}
\end{outerlist}

\halfblankline

\href{http://www.me.berkeley.edu/}{\textbf{Mechanical Engineering Department}}\\
\textbf{University of California$-$Berkeley}\\
Berkeley, California USA
\begin{outerlist}
\item[] \textit{Researcher}%
        \hfill \textbf{May 2006 to Aug 2007}
\begin{innerlist}
\item High sampling rate wireless sensor network design and development.
\item Designed system architecture, built sensor boards, implemented sensor software, implemented user interface.
\end{innerlist}
\end{outerlist}

%\halfblankline
%
%\href{http://www.mech.utah.edu/}{\textbf{Mechanical Engineering Department}}\\
%\textbf{University of Utah}\\
%Salt Lake City, Utah USA
%\begin{outerlist}
%\item[] \textit{Undergraduate Researcher}%
%        \hfill \textbf{Nov 2004 to May 2006}
%\begin{innerlist}
%\item Various projects: developed bio-inspired robot finger actuator, automated image processing, assisted in wind tunnel experiments, implemented software (Java, Matlab).
%\end{innerlist}
% FULL DETAIL %%
%\item[] \textit{Undergraduate Research Traineeship}%
%        \hfill \textbf{May 2005 to May 2006}
%\begin{innerlist}
%\item Research and develop anthropomorphic robotic finger actuator system.
%\item Derive system of equations to model finger tendon displacement joint kinematics.
%\item Develop visualization software in Java.
%\end{innerlist}
%
%%\halfblankline
%
%\item[] \textit{Undergraduate Researcher}%
%		\hfill \textbf{September 2005 to May 2006}\\
%		Advanced Fluid Dynamics Lab
%\begin{innerlist}
%\item Develop oil-film interferometry analysis software with \Matlab.
%\item Develop graphical software for removing noise and analyzing digital images.
%\end{innerlist}
%
%%\halfblankline
%
%\item[] \textit{Undergraduate Researcher}%
%		\hfill \textbf{January 2005 to August 2005}\\
%		Physical Fluid Dynamics Lab
%\begin{innerlist}
%\item Develop graphical data acquisition control software for fluid flow visualization experiments.
%\end{innerlist}
%
%%\halfblankline
%
%\item[] \textit{Undergraduate Research Assistantship}%
%		\hfill \textbf{November 2004 to August 2005}\\
%		Thermal-fluids Lab
%\begin{innerlist}
%\item Assist in experiments with transonic wind tunnel to determine aerodynamic losses due to effects of surface roughness with varying Mach number and turbulence. 
%\end{innerlist}
%\end{outerlist}



\section{Teaching Experience}
%
While at the University of California$-$Berkeley I was an instructor in classes for
%
\begin{innerlist}
    \item Vehicle dynamics and control
    \item Computer programming
    \item Dynamics
    \item Robotics
\end{innerlist}

%\section{Teaching Experience}
%%
%\textbf{University of California$-$Berkeley}\\
%Berkeley, California USA
%\begin{outerlist}
%\item[] \textit{Graduate Student Instructor}%
%        \hfill \textbf{Spring 2011}
%\begin{innerlist}
%\item Vehicle Dynamics and Control, Department of Mechanical Engineering.
%\item Automotive lateral and longitudinal dynamics; control methods for traction and stability, suspension systems, adaptive cruise control.
%\item Gave supplemental discussion section lectures, prepared homework assignments and solutions, maintained course website, advised and guided student group projects.
%\end{innerlist}
%
%\item[] \textit{Graduate Student Instructor}%
%		\hfill \textbf{Spring 2010}
%\begin{innerlist}
%\item Introduction to Programming, College of Engineering.
%\item Procedural and object-oriented programming, induction, iteration, recursion, functions, floating-point representations.
%\item Lead lab sections, prepared homework solutions, prepared exams.
%\end{innerlist}
%
%\item[] \textit{Graduate Student Lab Instructor}%
%		\hfill \textbf{Fall 2009}
%\begin{innerlist}
%\item Introduction to Robotics, Department of Electrical Engineering and Computer Science.
%\item Kinematics, dynamics, and control of robot manipulators; computer vision and perception.
%\item Prepared and set up lab projects, lead lab sections, advised and guided student group projects, assisted in exam review discussions.
%\end{innerlist}
%
%\item[] \textit{Graduate Student Instructor}%
%		\hfill \textbf{Fall 2009}
%\begin{innerlist}
%\item Intermediate Dynamics, Department of Mechanical Engineering.
%\item Kinematics and Newtonian dynamics of particles and rigid bodies.
%\item Gave supplemental discussion section lectures, prepared homework assignments and solutions, prepared exam questions and solutions, maintained course website.
%\end{innerlist}
%
%\end{outerlist}


\section{Academic Service}
%
\href{http://www.ieee.org}{\textbf{Institute of Electrical and Electronics Engineers}}
\begin{outerlist}
\item[] Journal Reviewer
		\hfill \textbf{2011 to present}
\begin{innerlist}
\item IEEE Transactions on Neural Networks and Learning Systems.
\end{innerlist}
\end{outerlist}


%\halfblankline

%\section{Academic Appointments}
%%
%\textbf{Postdoctoral Researcher} \hfill {September 2010 to present}
%\begin{innerlist}
%
%\item[] \href{http://www.cse.ohio-state.edu/}{Department of Computer Science and Engineering},\\
%        \href{http://www.osu.edu/}{The Ohio State University}
%\begin{innerlist}
%\item \href{http://www.nfs.gov/}{National Science Foundation} Cyber-Physical Systems (ENG, \href{http://www.nsf.gov/div/index.jsp?div=eccs}{ECCS})
%\begin{innerlist}
%\item[$-$] Autonomous Driving in Mixed-Traffic Urban Environments (grant~\href{http://www.nsf.gov/awardsearch/showAward.do?AwardNumber=0931669}{\#0931669})
%\item[$-$] Supervisor (co-PI):
%    \href{http://www.cse.ohio-state.edu/~paolo/}%
%         {Professor Paolo A.~G.~Sivilotti}
%\end{innerlist}
%\end{innerlist}
%
%\end{innerlist}

%\section{Societies \& Service}
%%
%Professional Societies
%\begin{innerlist}
%\item American Institute of Aeronautics and Astronautics~(AIAA)
%\item Institute for Electrical and Electronics Engineers~(IEEE)
%\end{innerlist}
%
%\halfblankline

%Honor Societies
%\begin{innerlist}
%\item Pi Tau Sigma (Mechanical Engineering, 2005--2006 Vice President, 2004--2005 Secretary)
%\item Beehive Honor Society
%\item Tau Beta Pi (Engineering)
%\item Pi Mu Epsilon (Mathematics)
%\end{innerlist}


%\section{Application Areas}
%%
%Autonomous Aircraft, Vehicles, and Systems; Cooperative Control of Autonomous Systems; Decentralized Estimation, Control, and Path Planning; Bayesian Estimation and Sensor Likelihood Function Design; Stationary and Mobile Wireless Sensor Networks.

%\section{Software \& Hardware Background}
%%
%Computer Programming and Simulation:
%%
%\begin{innerlist}
%    \item Java, C$++$, Objective-C, bash, Python, JavaScript, \Matlab, Simulink, LabVIEW.
%\end{innerlist}
%
%\halfblankline

%Web Techonologies:
%%
%\begin{innerlist}
%	\item Cassandra, Storm, Netty.
%\end{innerlist}
%
%\halfblankline

%Operating Systems:
%%
%\begin{innerlist}
%    \item Linux, UNIX, Apple, Windows.
%\end{innerlist}
%
%\halfblankline

%Instrumentation:
%\begin{innerlist}
%	\item National Instruments data acquisition hardware and software, bench-top equipment.
%\end{innerlist}
%
%\halfblankline

%\href{http://www.mathworks.com/products/matlab/}{\Matlab} skill set:
%%
%\begin{innerlist}
%    \item Differential and difference equation simulation; mobile network simulation;
%    	statistics, Kalman-like filtering, and Bayesian estimation;
%    	linear algebra; filter and controller design; Laplace, Z, and Fourier
%    	transforms; polynomials; nonlinear numerical methods; visualization and movies, etc.
%
%    \item Toolboxes: control system, filter design, signal processing,
%    	image processing, model predictive control, statistics, symbolic mathematics
%\end{innerlist}
%
%\halfblankline

%Version Control and Software Configuration Management:
%%
%\begin{innerlist}
%    \item SVN, CVS, and others
%\end{innerlist}
%

%\halfblankline

%Embedded and Real-time Systems:
%%
%\begin{innerlist}
%    \item Software development with \href{http://www.qnx.com/}{QNX Realtime Operating System}, \href{http://www.ni.com}{National Instruments} control hardware and software (e.g., RT, cRIO, CVI), Windows Real-time Target
%\end{innerlist}
%
%\halfblankline

%Instrumentation, Data Acquisition, and Measurement:
%%
%\begin{innerlist}
%    \item \href{http://www.mathworks.com/products/simulink/}{Simulink},
%        \href{http://www.ni.com/}{LabVIEW} and
%        \href{http://www.ni.com}{National Instruments}
%        data acquisition hardware and software,
%        bench-top equipment (e.g., Hewlett-Packard and Agilent scopes and meters)
%\end{innerlist}
%
%\halfblankline

%Productivity Applications:
%%
%\begin{innerlist}
%    \item \TeX{} (\LaTeX{}, \BibTeX{}), Word, Powerpoint, Excel, Vim, etc.
%\end{innerlist}
%
%\halfblankline

%Information Technology:
%%
%\begin{innerlist}
%    \item Networking (UDP, TCP), middleware (ICE), services (Apache, SQL)
%\end{innerlist}


%\section{Expertise}
%%
%% Control theory (open-loop; feedback; feedforward; linear (output feedback; state feedback; optimal steady-state, receding horizon, and model predictive) and nonlinear (sliding mode, adaptive)); probability, information, and estimation theory (Least Squares variants, Kalman filtering variants, particle filtering variants, Bayesian estimation, Maximum Likelihood Estimation); artificial intelligence, statistical and machine learning theory; optimization and 
%%
%Control Theory and Engineering:
%%
%\begin{innerlist}
%    \item Open-loop, closed-loop (output and state feedback), and feedforward control; linear control including PID, pole placement, and optimal steady-state (LQR), receding horizon, and model predictive; nonlinear control including variable structure systems, sliding-mode, and adaptive; dynamic optimization
%\end{innerlist}
%
%\halfblankline
%
%System Modeling and Analysis:
%%
%\begin{innerlist}
%    \item Rigid body dynamics, continuous-time system dynamics, discrete-time system dynamics (Markov), finite state automata, graphical modeling
%\end{innerlist}
%
%\halfblankline
%
%Signal Processing and Communications:
%%
%\begin{innerlist}
%    \item Probability, random variables, stochastic processes, estimation, information theory
%\end{innerlist}
%
%\halfblankline
%
%Statistical Learning Theory:
%%
%\begin{innerlist}
%	\item Maximum likelihood estimation, maximum a-posteriori estimation, system identification, least squares optimization, Bayesian parametric and nonparametric estimation and inference, graphical model inference, E-M algorithm, sampling theory, classification and clustering
%\end{innerlist}
%
%\halfblankline
%
%Artificial Intelligence and Reinforcement Learning:
%%
%\begin{innerlist}
%	\item Search; perceptrons, neural networks, and state vector machines; Markov decision processes and partially observable Markov decision processes; dynamic programming; Monte Carlo methods; value and policy iteration; Q-learning; path planning (probabilistic roadmaps, rapidly-exploring random tree, receding horizon information metric optimization); simultaneous localization and mapping
%\end{innerlist}
%
%\halfblankline
%
%Computer Vision:
%%
%\begin{innerlist}
%	\item Image filtering and smoothing; edge detection and orientation histogram; perspective projection and optical flow; subframe classification (e.g., by histogram of oriented gradients and state vector machine)
%\end{innerlist}


%\section{Publications}
%\begin{bibsection}
	\item Wood, J.G., and J.K.~Hedrick. Partition Learning for Multiagent Planning. \emph{Journal of Robotics}. Volume 2012, Article ID 590479. 2012. 
	
    \item Wood, J.G. Time Evolving Space Partitioning for Search and Tracking of an
		Unknown Number of Targets by a Team of Heterogeneous Autonomous Agents. Dissertation, 				University of California, Berkeley. 2011.
	
	\item Wood, J.G., and J.K.~Hedrick. Multi-agent Path Planning for an
    	Unknown Number of Targets over Dynamic Space Partitions.
        In: \emph{Proceedings of the 50th IEEE Conference on Decision
        and Control and European Control Conference (CDC-ECC~2011)},
        December 12--15, 2011.

    \item Wood, J.G., and J.K.~Hedrick. Space Partitioning and Classification
    	for Multi-target Search and Tracking by Heterogeneous Unmanned Aerial
    	System Teams. In: \emph{Proceedings of the 2011 AIAA Infotech@Aerospace
    	Conference}, March 28, 2011.
	
	\item Wood, J.G., B.~Kehoe, and J.K.~Hedrick. Target Estimate PDF-based
    	Optimal Path Planning Algorithm with Application to UAV Systems. In:
    	\emph{Proceedings of the 2010 ASME Dynamic Systems and Control Conference},
        September 13, 2010.

    \item Wood, J.G. Reliable Wireless Sensor Network for Data Acquisition. Thesis,
		University of California, Berkeley. 2008.
	
	\item Wood, J.G., and S.~Mascaro. Human Finger Muscle-Tendon System for Robotics.
		In: \emph{Utah Undergraduate Research Journal}, 6, pp. 75, 112. 2006.
	
	\item Garvey, J., B.~Kehoe, B.~Basso, M.~Godwin, J.~Wood, J.~Love, S.-Y.~Liu,
    	Z.~Kim, S.~Jackson, Y.~Fallah, T.~Fu, R.~Sengupta, and J.K.~Hedrick. An
    	Autonomous Unmanned Aerial Vehicle System for Sensing and Tracking. In:
    	\emph{Proceedings of the 2011 AIAA Infotech@Aerospace Conference},
        March 28, 2011.
    	
	\item Sengupta, R., J.~Connors, B.~Kehoe, Z.~Kim, T.~Kuhn, and J.~Wood.
        Final Report -- Autonomous Search and Rescue with ScanEagle. Prepared for
        Evergreen Unmanned Systems and Shell International Exploration and
        Production Inc., September, 2010.
\end{bibsection}


%\clearpage
%\section{References Available to Contact}
%%
\href{http://www.me.berkeley.edu/faculty/hedrick/}{\textbf{Dr.~J.~Karl Hedrick}}
(e-mail:~\href{mailto:khedrick@me.berkeley.edu}{khedrick@me.berkeley.edu}; phone:~+1-510-642-2482)
%
\begin{innerlist}
    \item James Marshall Wells Professor\\
        \href{http://vehicle.berkeley.edu/}{Vehicle Dynamics and Control Lab}\\
        Mechanical Engineering\\
        UC Berkeley
    \item[$\diamond$] 5104 Etcheverry Hall, Mailstop 1740, University of California$-$Berkeley, Berkeley, CA 94720-1740
    \item[$\star$] \emph{Dr.~Hedrick was my graduate adviser and PI at the Center for Collaborative Control of Unmanned Vehicles.}
\end{innerlist}

\halfblankline

\href{http://www.ce.berkeley.edu/faculty/faculty.php?id=262}{\textbf{Dr.~Raja Sengupta}}
(e-mail:~\href{mailto:sengupta@ce.berkeley.edu}{sengupta@ce.berkeley.edu}; phone:~+1-510-642-9540)
\begin{innerlist}
    \item Professor\\
    	\href{http://www.ce.berkeley.edu/programs/sys/}{Civil Systems Group}\\
        Civil and Environmental Engineering\\
        UC Berkeley
    \item[$\diamond$] 640 Sutardja Dai Hall, UC Berkeley, Berkeley, California 94720-1712 
    \item[$\star$] \emph{Dr.~Sengupta is a co-PI at the Center for Collaborative Control of Unmanned Vehicles.}
\end{innerlist}

\halfblankline

%\textbf{Jerod M.~Kendrick}
%(e-mail:~\href{mailto:Jerod.Kendrick@gmail.com}{Jerod.Kendrick@gmail.com}; phone:~+1-281-787-2367)
%\begin{innerlist}
%    \item Research Technician\\
%        Deepwater Flow Assurance\\
%        \href{http://www.shell.com/}{Shell Exploration \& Production}
%    \item[$\diamond$] 3737 Bellaire Blvd, Houston, TX 77025
%    \item[$\star$] \emph{Mr.~Kendrick oversaw a project I worked on to integrate autonomous search and rescue on a ScanEagle aircraft.}
%\end{innerlist}
%
%\halfblankline

\textbf{Dr.~Andrew Gray}
(e-mail:~\href{mailto:drewjgray@gmail.com}{drewjgray@gmail.com}; phone:~+1-510-878-0212)
\begin{innerlist}
	\item Manager\\
		Autopilot Group\\
		\href{http://www.teslamotors.com/}{Tesla Motors}
	\item[$\diamond$] 3500 Deer Creek, Palo Alto, CA 94304
	\item[$\star$] \emph{Dr.~Gray and I worked together on autonomous systems projects.}
\end{innerlist}


\end{document}

%%%%%%%%%%%%%%%%%%%%%%%%%% End CV Document %%%%%%%%%%%%%%%%%%%%%%%%%%%%%

%----------------------------------------------------------------------%
% The following is copyright and licensing information for
% redistribution of this LaTeX source code; it also includes a liability
% statement. If this source code is not being redistributed to others,
% it may be omitted. It has no effect on the function of the above code.
%----------------------------------------------------------------------%
% Copyright (c) 2007, 2008, 2009, 2010, 2011 by Theodore P. Pavlic
%
% Unless otherwise expressly stated, this work is licensed under the
% Creative Commons Attribution-Noncommercial 3.0 United States License. To
% view a copy of this license, visit
% http://creativecommons.org/licenses/by-nc/3.0/us/ or send a letter to
% Creative Commons, 171 Second Street, Suite 300, San Francisco,
% California, 94105, USA.
%
% THE SOFTWARE IS PROVIDED "AS IS", WITHOUT WARRANTY OF ANY KIND, EXPRESS
% OR IMPLIED, INCLUDING BUT NOT LIMITED TO THE WARRANTIES OF
% MERCHANTABILITY, FITNESS FOR A PARTICULAR PURPOSE AND NONINFRINGEMENT.
% IN NO EVENT SHALL THE AUTHORS OR COPYRIGHT HOLDERS BE LIABLE FOR ANY
% CLAIM, DAMAGES OR OTHER LIABILITY, WHETHER IN AN ACTION OF CONTRACT,
% TORT OR OTHERWISE, ARISING FROM, OUT OF OR IN CONNECTION WITH THE
% SOFTWARE OR THE USE OR OTHER DEALINGS IN THE SOFTWARE.
%----------------------------------------------------------------------%
