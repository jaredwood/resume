%%
%% Control theory (open-loop; feedback; feedforward; linear (output feedback; state feedback; optimal steady-state, receding horizon, and model predictive) and nonlinear (sliding mode, adaptive)); probability, information, and estimation theory (Least Squares variants, Kalman filtering variants, particle filtering variants, Bayesian estimation, Maximum Likelihood Estimation); artificial intelligence, statistical and machine learning theory; optimization and 
%%
%Control Theory and Engineering:
%%
\begin{loneinnerlist}
    \item \textbf{Control Theory}: open-loop, closed-loop (output and state feedback), and feedforward control; linear control including PID, pole placement, and optimal steady-state (LQR), receding horizon, and model predictive; nonlinear control including variable structure systems, sliding-mode, and adaptive; dynamic optimization
    \item \textbf{System Modeling}: rigid body dynamics, continuous-time system dynamics, discrete-time system dynamics (Markov), finite state automata, graphical modeling.
    \item \textbf{Signal Processing}: probability, random variables, stochastic processes, estimation, information theory
    \item Statistical learning: maximum likelihood estimation, maximum a-posteriori estimation, system identification, least squares optimization, Bayesian parametric and nonparametric estimation and inference, graphical model inference, E-M algorithm, sampling theory, classification and clustering
    \item \textbf{Artificial Intelligence}: search; perceptrons, neural networks, and state vector machines; Markov decision processes and partially observable Markov decision processes; dynamic programming; Monte Carlo methods; value and policy iteration; Q-learning; path planning (probabilistic roadmaps, rapidly-exploring random tree, receding horizon information metric optimization); simultaneous localization and mapping
    \item \textbf{Computer Vision}: image filtering and smoothing; edge detection and orientation histogram; perspective projection and optical flow; subframe classification (e.g., by histogram of oriented gradients and state vector machine)
\end{loneinnerlist}
