%%%%%%%%%%%%%%%%%%%%%%%%%%%%%%%%%%%%%%%%%%%%%%%%%%%%%%%%%%%%%%%%%%%%%%%%
%%%%%%%%%%%%%%%%%%%%%% Simple LaTeX CV Template %%%%%%%%%%%%%%%%%%%%%%%%
%%%%%%%%%%%%%%%%%%%%%%%%%%%%%%%%%%%%%%%%%%%%%%%%%%%%%%%%%%%%%%%%%%%%%%%%

%%%%%%%%%%%%%%%%%%%%%%%%%%%%%%%%%%%%%%%%%%%%%%%%%%%%%%%%%%%%%%%%%%%%%%%%
%% NOTE: If you find that it says                                     %%
%%                                                                    %%
%%                           1 of ??                                  %%
%%                                                                    %%
%% at the bottom of your first page, this means that the AUX file     %%
%% was not available when you ran LaTeX on this source. Simply RERUN  %%
%% LaTeX to get the ``??'' replaced with the number of the last page  %%
%% of the document. The AUX file will be generated on the first run   %%
%% of LaTeX and used on the second run to fill in all of the          %%
%% references.                                                        %%
%%%%%%%%%%%%%%%%%%%%%%%%%%%%%%%%%%%%%%%%%%%%%%%%%%%%%%%%%%%%%%%%%%%%%%%%

%%%%%%%%%%%%%%%%%%%%%%%%%%%% Document Setup %%%%%%%%%%%%%%%%%%%%%%%%%%%%

% Don't like 10pt? Try 11pt or 12pt
\documentclass[10pt]{article}

% The automated optical recognition software used to digitize resume
% information works best with fonts that do not have serifs. This
% command uses a sans serif font throughout. Uncomment both lines (or at
% least the second) to restore a Roman font (i.e., a font with serifs).
\usepackage{times}
\renewcommand{\familydefault}{\sfdefault}

% The OCR software also has a hard time with italics. These commands get
% rid of the two common ways to italicize text in LaTeX. Get rid of them
% to turn italics back on.
\renewcommand\emph[1]{#1}
\renewcommand\textit[1]{#1}

% This is a helpful package that puts math inside length specifications
\usepackage{calc}

% Simpler bibsection for CV sections
% (thanks to natbib for inspiration)
\makeatletter
\newlength{\bibhang}
\setlength{\bibhang}{1em}
\newlength{\bibsep}
 {\@listi \global\bibsep\itemsep \global\advance\bibsep by\parsep}
\newenvironment{bibsection}%
        {\vspace{-\baselineskip}\begin{list}{}{%
       \setlength{\leftmargin}{\bibhang}%
       \setlength{\itemindent}{-\leftmargin}%
       \setlength{\itemsep}{\bibsep}%
       \setlength{\parsep}{\z@}%
        \setlength{\partopsep}{0pt}%
        \setlength{\topsep}{0pt}}}
        {\end{list}\vspace{-.6\baselineskip}}
\makeatother

% Layout: Puts the section titles on left side of page
\reversemarginpar

%
%         PAPER SIZE, PAGE NUMBER, AND DOCUMENT LAYOUT NOTES:
%
% The next \usepackage line changes the layout for CV style section
% headings as marginal notes. It also sets up the paper size as either
% letter or A4. By default, letter was used. If A4 paper is desired,
% comment out the letterpaper lines and uncomment the a4paper lines.
%
% As you can see, the margin widths and section title widths can be
% easily adjusted.
%
% ALSO: Notice that the includefoot option can be commented OUT in order
% to put the PAGE NUMBER *IN* the bottom margin. This will make the
% effective text area larger.
%
% IF YOU WISH TO REMOVE THE ``of LASTPAGE'' next to each page number,
% see the note about the +LP and -LP lines below. Comment out the +LP
% and uncomment the -LP.
%
% IF YOU WISH TO REMOVE PAGE NUMBERS, be sure that the includefoot line
% is uncommented and ALSO uncomment the \pagestyle{empty} a few lines
% below.
%

%% Use these lines for letter-sized paper
\usepackage[paper=letterpaper,
            %includefoot, % Uncomment to put page number above margin
            marginparwidth=1.2in,     % Length of section titles
            marginparsep=.05in,       % Space between titles and text
            margin=1in,               % 1 inch margins
            includemp]{geometry}

%% Use these lines for A4-sized paper
%\usepackage[paper=a4paper,
%            %includefoot, % Uncomment to put page number above margin
%            marginparwidth=30.5mm,    % Length of section titles
%            marginparsep=1.5mm,       % Space between titles and text
%            margin=25mm,              % 25mm margins
%            includemp]{geometry}

%% More layout: Get rid of indenting throughout entire document
\setlength{\parindent}{0in}

%% This gives us fun enumeration environments. compactitem will be nice.
\usepackage{paralist}

%%% Setup header and footer (with page number and possible last page)
%
% The first block sets up pages 2--end
% The second block sets up page 1 formatting
%
%%%
%
% NOTE: comment the +LP lines and uncomment the -LP lines to have page
%       numbers without the ``of ##'' last page reference)
%
% NOTE: uncomment the \pagestyle{empty} line to get rid of all page
%       numbers on pages 2--end. To get rid of page numbers on page 1,
%       comment out the \thispagestyle{plain} line on the first page
%       below.
%       (also make sure includefoot is commented out above)
%
\usepackage{fancyhdr,lastpage}
\pagestyle{fancy}
%\pagestyle{empty}      % Uncomment this to get rid of page numbers
\fancyhf{}\renewcommand{\headrulewidth}{0pt}
\fancyfootoffset{\marginparsep+\marginparwidth}
\newlength{\footpageshift}
\setlength{\footpageshift}
          {0.5\textwidth+0.5\marginparsep+0.5\marginparwidth-2in}

%%%% PAGES 2--9 NUMBERING:
%% These two lines put page number in upper-right corner of pages 2--end
\rhead{Pavlic, p.~\arabic{page} of \protect\pageref*{LastPage}}   % +LP
%\rhead{Pavlic, p.~\arabic{page}}                                 % -LP

%% These lines put page number in bottom (center) of pages 2--end
%\lfoot{\hspace{\footpageshift}%
%       \parbox{4in}{\, \hfill %
%                    \arabic{page} of \protect\pageref*{LastPage} % +LP
%%                    \arabic{page}                               % -LP
%                    \hfill \,}}
%%%% END PAGE 2--9 NUMBERING

%%%% PAGE 1 NUMBERING:
\makeatletter
\let\oldps@plain\ps@plain
\renewcommand{\ps@plain}{\oldps@plain%
\renewcommand{\@evenfoot}{\hfil %
    p.~\arabic{page} of \protect\pageref*{LastPage} % +LP
%    p.~\arabic{page}                               % -LP
    \hfil}%
\renewcommand{\@oddfoot}{\@evenfoot}}
\makeatother
%%%% END PAGE 1 NUMBERING

% Finally, give us PDF bookmarks and colored links
%
% NOTE: Some OCR software might be negatively affected by hyperlinks. So
%       most employers recommend the draft option here.
%
% (to enable hyperlinks and bookmarks, comment out ``draft'' line;
%  to disable hyperlinks and bookmarks, uncomment ``draft'' line)
\usepackage{color,hyperref}
\definecolor{darkblue}{rgb}{0.0,0.0,0.3}
\hypersetup{colorlinks,breaklinks,
            linkcolor=darkblue,urlcolor=darkblue,
            anchorcolor=darkblue,citecolor=darkblue,
            draft
            }

%%%%%%%%%%%%%%%%%%%%%%%% End Document Setup %%%%%%%%%%%%%%%%%%%%%%%%%%%%


%%%%%%%%%%%%%%%%%%%%%%%%%%% Helper Commands %%%%%%%%%%%%%%%%%%%%%%%%%%%%

% The title (name) with a horizontal rule under it
% (optional argument typesets an object right-justified across from name
%  as well)
%
% Usage: \makeheading{name}
%        OR
%        \makeheading[right_object]{name}
%
% Place at top of document. It should be the first thing.
% If ``right_object'' is provided in the square-braced optional
% argument, it will be right justified on the same line as ``name'' at
% the top of the CV. For example:
%
%       \makeheading[\emph{Curriculum vitae}]{Your Name}
%
% will put an emphasized ``Curriculum vitae'' at the top of the document
% as a title. Likewise, a picture could be included:
%
%   \makeheading[\includegraphics[height=1.5in]{my_picutre}]{Your Name}
%
% the picture will be flush right across from the name.
\newcommand{\makeheading}[2][]%
        {\hspace*{-\marginparsep minus \marginparwidth}%
         \begin{minipage}[t]{\textwidth+\marginparwidth+\marginparsep}%
             {\large \bfseries #2 \hfill #1}\\[-0.15\baselineskip]%
                 \rule{\columnwidth}{1pt}%
         \end{minipage}}

% The section headings
%
% Usage: \section{section name}
%
% Follow this section IMMEDIATELY with the first line of the section
% text. Do not put whitespace in between. That is, do this:
%
%       \section{My Information}
%       Here is my information.
%
% and NOT this:
%
%       \section{My Information}
%
%       Here is my information.
%
% Otherwise the top of the section header will not line up with the top
% of the section. Of course, using a single comment character (%) on
% empty lines allows for the function of the first example with the
% readability of the second example.
\renewcommand{\section}[2]%
        {\pagebreak[3]\vspace{1.3\baselineskip}%
         \phantomsection\addcontentsline{toc}{section}{#1}%
         \hspace{0in}%
         \marginpar{
         \raggedright \scshape #1}#2}

% An itemize-style list with lots of space between items
\newenvironment{outerlist}[1][\enskip\textbullet]%
        {\begin{itemize}[#1]}{\end{itemize}%
         \vspace{-.6\baselineskip}}

% An environment IDENTICAL to outerlist that has better pre-list spacing
% when used as the first thing in a \section
\newenvironment{lonelist}[1][\enskip\textbullet]%
        {\vspace{-\baselineskip}\begin{list}{#1}{%
        \setlength{\partopsep}{0pt}%
        \setlength{\topsep}{0pt}}}
        {\end{list}\vspace{-.6\baselineskip}}

% An itemize-style list with little space between items
\newenvironment{innerlist}[1][\enskip\textbullet]%
        {\begin{compactitem}[#1]}{\end{compactitem}}

% An environment IDENTICAL to innerlist that has better pre-list spacing
% when used as the first thing in a \section
\newenvironment{loneinnerlist}[1][\enskip\textbullet]%
        {\vspace{-\baselineskip}\begin{compactitem}[#1]}
        {\end{compactitem}\vspace{-.6\baselineskip}}

% To add some paragraph space between lines.
% This also tells LaTeX to preferably break a page on one of these gaps
% if there is a needed pagebreak nearby.
\newcommand{\blankline}{\quad\pagebreak[3]}
\newcommand{\halfblankline}{\quad\vspace{-0.5\baselineskip}\pagebreak[3]}

% Uses hyperref to link DOI
\newcommand\doilink[1]{\href{http://dx.doi.org/#1}{#1}}
\newcommand\doi[1]{doi:\doilink{#1}}

% For \url{SOME_URL}, links SOME_URL to the url SOME_URL
\providecommand*\url[1]{\href{#1}{#1}}
% Same as above, but pretty-prints SOME_URL in teletype fixed-width font
\renewcommand*\url[1]{\href{#1}{\texttt{#1}}}

% For \email{ADDRESS}, links ADDRESS to the url mailto:ADDRESS
\providecommand*\email[1]{\href{mailto:#1}{#1}}
% Same as above, but pretty-prints ADDRESS in teletype fixed-width font
%\renewcommand*\email[1]{\href{mailto:#1}{\texttt{#1}}}

%\providecommand\BibTeX{{\rm B\kern-.05em{\sc i\kern-.025em b}\kern-.08em
%    T\kern-.1667em\lower.7ex\hbox{E}\kern-.125emX}}
%\providecommand\BibTeX{{\rm B\kern-.05em{\sc i\kern-.025em b}\kern-.08em
%    \TeX}}
\providecommand\BibTeX{{B\kern-.05em{\sc i\kern-.025em b}\kern-.08em
    \TeX}}
\providecommand\Matlab{\textsc{Matlab}}

%%%%%%%%%%%%%%%%%%%%%%%% End Helper Commands %%%%%%%%%%%%%%%%%%%%%%%%%%%

%%%%%%%%%%%%%%%%%%%%%%%%% Begin CV Document %%%%%%%%%%%%%%%%%%%%%%%%%%%%

\begin{document}
\thispagestyle{plain}
\makeheading[\emph{Curriculum vitae}]{Theodore~(Ted) P.~Pavlic}

\section{Contact Information}
%
% NOTE: Mind where the & separators and \\ breaks are in the following
%       table.
%
% ALSO: \rcollength is the width of the right column of the table
%       (adjust it to your liking; default is 1.85in).
%
\newlength{\rcollength}\setlength{\rcollength}{1.85in}%
%
\begin{tabular}[t]{@{}p{\textwidth-\rcollength}p{\rcollength}}
\href{http://www.cse.osu.edu/}%
     {Department of Computer Science and Engineering} & \\
\href{http://www.osu.edu/}{The Ohio State University}
                           & \textit{Mobile:} +1-760-483-3390 \\
395 Dreese Labs            & \textit{Fax:} +1-614-292-2911 \\
2015 Neil Avenue           & \textit{E-mail:} \email{pavlic.3@osu.edu}\\
Columbus, OH  43210 USA    & \textit{WWW:}
\href{http://www.tedpavlic.com/}{www.tedpavlic.com}\\
\end{tabular}

\section{Objective}
%
Full-time position that allows for advanced research in electrical and
computer engineering (communications, control, software, and
electronics), with a particular focus on complex distributed systems
(i.e., modeling, analysis, design, and verification)
\begin{innerlist}
    \item For more information, see \url{http://www.tedpavlic.com/engjobsearch/}
\end{innerlist}

\section{Availability}%
\begin{loneinnerlist}
    \item Start time is negotiable, but preferably July--August 2011
    \item Geographic location is flexible, but preferably near one of:
        Austin, TX; Boston, MA; Los Angeles, CA; Salt Lake City, UT;
        Tempe, AZ
\end{loneinnerlist}

\section{Security Clearance}
%
Department of Defense Top Secret SCI with polygraph (expired: 2002)

% \section{Citizenship}
% %
% USA

\section{Research Interests}
%
Agent-based modeling, complex systems, hybrid systems, distributed
algorithms, amorphous computing, autonomous systems and vehicles,
networks, control theory, communication theory, behavioral ecology,
cooperation theory, optimization, parallel computation, robotics,
bio-mimicry and bio-inspiration

\section{Education}
%
\href{http://www.osu.edu/}{\textbf{The Ohio State University}},
Columbus, Ohio USA
\begin{outerlist}

\item[] Ph.D.,
        \href{http://www.ece.osu.edu/}
             {Electrical and Computer Engineering}, August 2010
             \hfill GPA: 3.70 (4.0 scale)
        \begin{innerlist}
        \item Thesis Topic: \emph{Design and Analysis of Optimal
            Task-Processing Agents}
        \item Thesis Proposal: \emph{Cooperative Task Processing}
        \item Candidacy: \emph{Research
            Problems in Distributed Control for Energy Systems}
        \item Adviser:
              \href{http://www.ece.osu.edu/~passino/}
                   {Professor Kevin M.~Passino}
        \item Area of Study: Control Engineering
        \end{innerlist}

\item[] M.S.,
        \href{http://www.ece.osu.edu/}
             {Electrical and Computer Engineering}, August 2007
             \hfill GPA: 3.70 (4.0 scale)
        \begin{innerlist}
        \item Thesis Topic: \emph{Optimal Foraging Theory Revisited}
        \item Adviser:
              \href{http://www.ece.osu.edu/~passino/}
                   {Professor Kevin M.~Passino}
        \item Area of Study: Control Engineering
        \end{innerlist}

\item[] B.S.,
        \href{http://www.ece.osu.edu/}
             {Electrical and Computer Engineering}, June 2004
             \hfill GPA: 3.86 (4.0 scale)
        \begin{innerlist}
        \item \emph{Magna cum Laude}, With Honors in Engineering
        \item Electrical specialization (emphasis on electromagnetics and digital computers)
        \item Minor in \href{http://www.cse.ohio-state.edu/}
                            {Computer and Information Systems}
              (programming and algorithms)
        \end{innerlist}

\end{outerlist}

\section{Academic Appointments}
%
\textbf{Postdoctoral Researcher} \hfill {September 2010~-- present}
\begin{innerlist}

\item[] \href{http://www.cse.ohio-state.edu/}{Department of Computer Science and Engineering},\\
        \href{http://www.osu.edu/}{The Ohio State University}
\begin{innerlist}
\item \href{http://www.nfs.gov/}{National Science Foundation} Cyber-Physical Systems (ENG, \href{http://www.nsf.gov/div/index.jsp?div=eccs}{ECCS})
\begin{innerlist}
\item[$-$] Autonomous Driving in Mixed-Traffic Urban Environments (grant~\href{http://www.nsf.gov/awardsearch/showAward.do?AwardNumber=0931669}{\#0931669})
\item[$-$] Supervisor (co-PI):
    \href{http://www.cse.ohio-state.edu/~paolo/}%
         {Professor Paolo A.~G.~Sivilotti}
\end{innerlist}
\end{innerlist}

\end{innerlist}

\section{Refereed\\Journal\\Publications} \begin{bibsection}
    \item Pavlic, T.P., and K.M.~Passino. Generalizing foraging theory
        for analysis and design. \emph{The International Journal of
        Robotics Research [Special Issue on Stochasticity in Robotics
        and Bio-Systems Part 1]}. 30(5):505--523. 2011.\\
        \doi{10.1177/0278364910396551}

    \item Pavlic, T.P., and K.M.~Passino. The sunk-cost effect as an
        optimal rate-maximizing behavior. \emph{Acta Biotheoretica},
        59(1):53--66. 2011.\\
        \doi{10.1007/s10441-010-9107-8}

    \item Pavlic, T.P., and K.M.~Passino. When rate maximization is
        impulsive. \emph{Behavioral Ecology and Sociobiology},
        64(8):1255--1265. August 2010.\\
        \doi{10.1007/s00265-010-0940-1}

    \item Pavlic, T.P., and K.M.~Passino. Foraging theory for autonomous
        vehicle speed choice. \emph{Engineering Applications of
        Artificial Intelligence}, 22(3):482--489, April
        2009.\\
        \doi{10.1016/j.engappai.2008.10.017}
\end{bibsection}

\section{Submitted\\Journal\\Publications} \begin{bibsection}
    \item Pavlic, T.P., and K.M.~Passino. Cooperative task
        processing. \emph{IEEE Transactions of Automatic Control}. 2010.
        Submitted.
\end{bibsection}

% Add a little space to nudge next ``Conference Publications'' marginpar
% down to make room for tall ``Submitted Journal Publications''
% marginpar. If there are enough submitted journal publications, this
% space will not be needed (and should be removed).
\vspace{0.1in}

\section{Conference\\Publications} \begin{bibsection}
    \item Pavlic, T.P. Stigmergic memory for real-time primal-space
        distributed optimization under constraints.
        In: \emph{Proceedings of the 50th IEEE Conference on Decision
        and Control and European Control Conference (CDC-ECC~2011)},
        December 12--15, 2011. Submitted.

    \item Pavlic, T.P., and K.M.~Passino. Cooperative task-processing
        networks. In: \emph{Proceedings of the Second International
        Workshop on Networks of Cooperating Objects (CONET~2011)},
        April 11, 2011.

    \item Pavlic, T.P., and K.M.~Passino. Cooperative task
        processing. In: \emph{Proceedings of the ICAM 2009 Symposium:
        Emergence in Physical, Biological, and Social Systems IV},
        November 13, 2009. Poster abstract.

    \item Freuler, R.J., M.J.~Hoffmann, T.P.~Pavlic, J.M.~Beams,
        J.P.~Radigan, P.K.~Dutta, J.T.~Demel, and E.D.~Justen.
        Experiences with a Comprehensive Freshman Hands-On Course~--
        Designing, Building, and Testing Small Autonomous Robots. In:
        \emph{Proceedings of the 2003 American Society for Engineering
        Education Annual Conference \& Exposition}, 2003.
\end{bibsection}

\section{Other\\Publications} \begin{bibsection}
    \item Pavlic, T.P., and K.M.~Passino. Cooperative Task-processing
        Networks: Parallel Computation of Non-trivial Volunteering
        Equilibria. Tech.~report OSU-CISRC-3/11-TR05, The Ohio State
        University, 2011.

    \item Pavlic, T.P. \emph{Design and Analysis of Optimal
        Task-Processing Agents}. PhD thesis, The Ohio State University,
        Columbus, OH, 2010.

    \item Pavlic, T.P. \emph{Optimal Foraging Theory Revisited}.
        Master's thesis, The Ohio State University, Columbus, OH, 2007.
\end{bibsection}

\section{Books in Preparation} \begin{bibsection}
    \item Pavlic, T.P., B.W.~Andrews, K.M.~Passino, and T.A.~Waite.
        \emph{Foraging Theory for Engineering}.
\end{bibsection}

\section{Papers in Preparation} \begin{bibsection}
    \item Pavlic, T.P., K.M.~Passino. Distributed optimization under
        constraints: Pareto-optimal intelligent lighting.

    \item Pavlic, T.P. The ideal free distribution as degenerate form of
        nutrient-constrained optimization.
\end{bibsection}

\section{Referee Service} \begin{loneinnerlist}
    \item \emph{49\textsuperscript{th} Annual Conference on Decision and Control}
    \item \emph{Bioinspiration \& Biomimetics}
    \item \emph{Behavioral Ecology}
    \item \emph{IEEE Transactions on Signal Processing}
    \item \emph{The International Journal of Robotics Research}
    \item \emph{Swarm and Evolutionary Computation}
    \item \emph{IEEE Transactions on Control Systems Technology}
\end{loneinnerlist}

\section{Conference Service} \begin{bibsection}
    \item Organizer/Associate Editor for invited session: ``Correctness
        by Verification and Design'', 14\textsuperscript{th} IEEE
        Conference on Intelligent Transportation Systems~(ITSC~2011),
        Washington, DC, October 5--7, 2011 (submitted)
\end{bibsection}

\section{Professional Experience}
%
\href{http://www.ni.com/}{\textbf{National Instruments}},
Austin, Texas USA
\begin{outerlist}

\item[] \textit{Analog Hardware R\&D Intern for Multifunction DAQ}%
        \hfill \textbf{June--September 2003}
\begin{innerlist}
\item Designed final verification testing fixture for use with STC2 MIO
        products.
\item Designed and executed study of the effect of varying burn-in time
        on long-term drift of common industry voltage references.
\end{innerlist}

\item[] \textit{Analog Hardware R\&D Intern for Multifunction DAQ}%
        \hfill \textbf{June--September 2002}
\begin{innerlist}
\item Designed and performed validation tests on new 16-bit 800 kHz
        NI-6120 SMIO DAQ board.

\item Designed high quality filter/amplifier source for use with NI-5411
        arbitrary function generator.
\end{innerlist}

\end{outerlist}

\halfblankline

\textbf{\href{http://www.ibm.com/}{IBM} Network Storage},
Research Triangle Park, North Carolina USA
\begin{outerlist}

\item[] \textit{Core Systems Software Developer for FlexNAS}%
        \hfill \textbf{June--September 2001}
\begin{innerlist}
\item Designed and implemented high-availability, redundant multihop
        communications subsystem.
\item Participated in software development of various vital box
        services.
\end{innerlist}

\end{outerlist}

\halfblankline

\href{http://www.calltech.com/}{\textbf{CallTech Communications}},
Columbus, Ohio USA
\begin{outerlist}

\item[] \textit{Information Technology Systems Engineer}%
        \hfill \textbf{June 1997~-- May 2001}
\begin{innerlist}
\item Responsible for the acquisition, setup, maintenance, and
        administration of all Internet hardware and software supporting
        \href{http://www.netwalk.com/}{NetWalk} Internet service
        and web presence provider.
\item Designed and implemented state of the art open source
        high-availability load balancing system supporting thousands of
        virtual servers.
\item Developed software call center support software for clients such
        as CompuServe, AOL, and Priceline.
\end{innerlist}

\end{outerlist}

\halfblankline

\textbf{MegaLinx Communications}, Dublin, Ohio USA
\begin{outerlist}

\item[] \textit{Web Developer and Support Representative}%
        \hfill \textbf{June 1995~-- May 1997}
\begin{innerlist}
\item Produced web content for commercial clients.
\item Assisted in administration of UltraSPARC, x86, 68020, 68030, and
        PowerPC systems running Sun Solaris, Linux, Microsoft DOS,
        Microsoft Windows NT, and Apple Macintosh operating systems.
\item Developed multi-platform open source file sharing solution.
\item Provided technical support for Internet and web presence
        customers.
\end{innerlist}

\end{outerlist}

\section{Teaching Experience}
\href{http://www.osu.edu}{\textbf{The Ohio State University}},
Columbus, Ohio USA
\begin{outerlist}

\item[] \textit{Teaching Assistant}%
    \hfill \textbf{September 2007~-- August 2009}\\
    (sample graded material and student evaluations available upon
    request)
    \begin{innerlist}
        \item Instructor for ECE~327: Electronic Devices and Circuits Laboratory I
        \begin{innerlist}
            \item Autumn~2007, Winter~2008 (2 sections), Spring~2008
                (2 sections), Winter~2009 (2 sections), and Summer~2009

            \item Sample student evaluations available upon request

            \item Responsible for 1~hour lecture and supervision of
                3~hour laboratory where junior and senior undergraduate
                students design and implement infrared modem and speaker
                driver for analog electronic audio signals

            \item Developed hundreds of pages of supplementary course
                material, including a course web page archived at\\
                \url{http://www.tedpavlic.com/teaching/osu/ece327}
        \end{innerlist}

        \halfblankline

        \item Grader for ECE~481 Ethics in Electrical and Computer Engineering
        \begin{innerlist}
            \item Autumn~2007 and Autumn~2008
        \end{innerlist}

        \halfblankline

        \item Instructor for ECE~209: Circuits and Electronics
            Laboratory
        \begin{innerlist}
            \item Autumn~2008

            \item Sample student evaluations available upon request

            \item Responsible for 0.5~hour lecture and supervision of
                3.5~hour laboratory where sophomore undergraduate
                students learn learn how to use basic laboratory
                equipment to study properties of electronic circuits

            \item Developed supplementary course
                material, including a course web page archived at
                \url{http://www.tedpavlic.com/teaching/osu/ece209}
        \end{innerlist}

        \halfblankline

        \item Instructor for ECE~557: Control, Signals, and Systems
            Laboratory
        \begin{innerlist}
            \item Summer~2008 (2 sections) and Summer~2009

            \item Sample student evaluations available upon request

            \item Responsible for 0.5~hour lecture and supervision of
                3.5~hour laboratory where senior undergraduate students
                combine
                \href{http://www.mathworks.com/products/simulink/}{Simulink},
                with \href{http://www.dspaceinc.com/}{dSPACE} RTI1104
                real-time control hardware and software to do analysis
                and control implementation for linear systems

            \item Developed supplementary course
                material, including a course web page archived at
                \url{http://www.tedpavlic.com/teaching/osu/ece557}
        \end{innerlist}

        \halfblankline

        \item Lab Instructor for ECE~758: Control Systems Implementation
            Laboratory
        \begin{innerlist}
            \item Spring~2009 (2 sections)

            \item Sample student evaluations available upon request

            \item Responsible for 0.5~hour lecture and supervision of
                3.5~hour laboratory where graduate students and senior
                undergraduate students combine
                \href{http://www.mathworks.com/products/simulink/}{Simulink},
                with \href{http://www.dspaceinc.com/}{dSPACE} RTI1104
                real-time control hardware and software to do analysis
                and advanced control implementation for linear and
                non-linear systems

            \item Developed supplementary course
                material, including a course web page archived at
                \url{http://www.tedpavlic.com/teaching/osu/ece758}
        \end{innerlist}
    \end{innerlist}

\item[] \href{http://www.nsfgk12.org/}
        {\emph{National Science Foundation GK-12 Fellow}}
        \hfill \textbf{September 2006~-- October 2007}
\begin{innerlist}
    \item[] Developed, implemented, and evaluated daily inquiry-based
        fourth-grade science lessons for a local inner-city public
        school class.
\end{innerlist}

\item[] \textit{Instructor}%
        \hfill \textbf{March 2002~-- June 2004}
\begin{innerlist}
\item Member of \href{http://feh.eng.ohio-state.edu/}
                     {Fundamentals of Engineering for Honors}
      instructional team.
\item Special graduate teaching appointment as undergraduate.
\item Lectured weekly engineering laboratory for ENG~H191,
        H192, and~H193.
\item Trained in-class undergraduate teaching assistants in laboratory
        procedure.
\item Graded weekly lab reports and provided laboratory exams.
\end{innerlist}

\item[] \textit{Teaching Assistant}%
        \hfill \textbf{September 2000~-- March 2002}
\begin{innerlist}
\item Assisted \href{http://feh.eng.ohio-state.edu/}
                    {Fundamentals of Engineering for Honors}
      instructional team.
\item Provided in-class support to first-year engineering students
        (ENG~H191, H192, and~H193).
\item Graded daily assignments on programming and drafting.
\item Developed on-line journal submission and report system for Physics
        Education Research Group~(PERG).
\end{innerlist}

\item[] \textit{Undergraduate Researcher}%
        \hfill \textbf{September 2000~-- March 2002}
\begin{innerlist}
\item Participated in the
        \href{http://www.cse.ohio-state.edu/europa/}{Europa
        Undergraduate Research Forum}, a part of the
        \href{http://www.cse.ohio-state.edu/rsrg/}{Reusable Software
        Research Group}.
\item Studied component-based software engineering undergraduate
        pedagogy.
\item Researched changes to RESOLVE/C++ implementation for ANSI
        compliance.
\end{innerlist}

\item[] \textit{Grader}%
        \hfill \textbf{September--December 2001}
\begin{innerlist}
\item Graded daily electromagnetics assignments (ECE~311).
\end{innerlist}
\end{outerlist}

\section{Service}
%
Recent contributor to several open-source software projects, including:
\begin{innerlist}
    \item \href{http://vim-latex.sourceforge.net/}{Vim-LaTeX} suite
    \item \href{http://vimperator.org}{Vimperator} and
        \href{http://dactyl.sourceforge.net/pentadactyl/index}{Pentadactyl}
        Firefox extensions
    \item \href{http://git-scm.com}{Git} distributed version control
        system
    \item \href{http://www.selenic.com/mercurial/}{Mercurial} distributed version control
        system
    \item Personal projects archived at
        \url{http://hg.tedpavlic.com/}
\end{innerlist}

\halfblankline

Frequent contributor to \href{http://www.wikipedia.org/}{Wikipedia}.
%
\begin{innerlist}
    \item Significant contributions to articles on control theory,
        electronics, and signals and systems.
\end{innerlist}

\halfblankline

\href{http://www.osufirst.org/}{OSU FIRST Robotics Team},
\href{http://www.osu.edu}{The Ohio State University}, 2000--2004
\begin{innerlist}
\item Introduced middle school and high school students to science and
        technology by participating with them in national robotics
        competitions.
\item Led 2002 team to regional silver medal
        \href{http://www.firstwiki.org/Engineering_Inspiration_Award}
             {\emph{Engineering Inspiration Award}}.
\item \emph{Lead Team Mentor}, 2002--2004
\item \emph{Component Design Team Lead Mentor}, 2001--2002
\end{innerlist}

\halfblankline

Institute for Electrical and Electronics Engineers~(IEEE), Member,
2002--present
%
\begin{innerlist}
\item IEEE Control Systems Society (2004--present)
\item IEEE Computer Society (2009--present)
\item IEEE Intelligent Transportation Systems Society (2011--present)
\item IEEE Systems, Man, and Cybernetics Society (2011--present)
\item IEEE Robotics and Automation Society (2011--present)
\end{innerlist}

\halfblankline

Animal Behavior Society, Member, 2011--present

\halfblankline

Director of Computers,
\href{http://ec.osu.edu/}{Engineers' Council},
\href{http://www.osu.edu/}{The Ohio State University}, 2002

\halfblankline

\href{http://www.linuxvirtualserver.org/}
     {Linux Virtual Server Project}, 1999--2000
\begin{innerlist}
\item Early member of the team that formed the open-source project that
        is now an important load balancing solution for the Linux
        software platform.
\end{innerlist}

\halfblankline

\href{http://www.gcfn.org/}
     {Greater Columbus Free-Net}, 1995--1997
\begin{innerlist}
\item Provided technical support services.
\end{innerlist}

\halfblankline

CompuTeen Bulletin Board System, 1993--1995
\begin{innerlist}
\item Administrated dial-up bulletin board system.
\item Founded and administrated TeenLiNK, an international electronic
        mail network that spread through the United States, Canada, and
        Australia and delivered mail over a series of electronic dial-up
        drop offs.
\end{innerlist}


\section{Awards}
%
\href{http://www.nsf.gov/}{National Science Foundation}
\begin{innerlist}
\item \href{http://www.nsfgk12.org/}{GK-12 Fellowship}, 2006--2007
\item \href{http://www.nsf.gov/grfp}
           {Graduate Research Fellowship} Honorable Mention, 2005
\end{innerlist}

\halfblankline

\href{http://www.osu.edu}{The Ohio State University}
\begin{innerlist}
\item \href{http://www.gradsch.osu.edu/Content.aspx?Content=44&itemid=2}
           {Dean's Distinguished University Fellowship}, 2004--2010
\item Electrical and Computer Engineering Bradshaw Scholarship,
        2002--2004
\item Electrical and Computer Engineering Shafstall Scholarship,
        2001--2003
\item University Scholarship, 1999--2003
\end{innerlist}

\section{Application Areas}
%
Autonomous/Unmanned Vehicles, Flexible Manufacturing Systems,
Distributed Power Generation, Intelligent Lighting, Power Demand
Response, Microgrids, Smart Grids

\section{Hardware and Software Skills}
%
Analog and Digital Electronics:
%
\begin{innerlist}
    \item Bipolar and FET implementations of continuous and switched
        amplifiers, modulators, converters, and filters

    \item Computer-Aided Design Tools: Cadence OrCAD, NI Multisim, SPICE, pst-circ
\end{innerlist}

\halfblankline

Embedded and Real-time Systems:
%
\begin{innerlist}
    \item Software and hardware development with several MCU and
        DSP platforms (e.g., Motorola MCU's, Texas Instruments DSP's, Atmel
        ATmega MCU's, Microchip PIC MCU's, and others)
\end{innerlist}

\halfblankline

Instrumentation, Control, Data Acquisition, Test, and Measurement:
%
\begin{innerlist}
    \item \href{http://www.dspaceinc.com/}{dSPACE} hardware (e.g.,
        RTI1104) and Control Desk software,
        \href{http://www.mathworks.com/products/simulink/}{Simulink},
        \href{http://www.ni.com/}{LabVIEW} and other
        \href{http://www.ni.com}{National Instruments}
        control and data acquisition hardware and software (e.g., MIO,
        SMIO, DSA, DMM, and others), Hewlett-Packard and Agilent
        bench-top equipment
\end{innerlist}

\halfblankline

Computer Programming:
%
\begin{innerlist}
    \item C, C$+$$+$, Java, JavaScript, Pascal, Perl, PHP, Lisp, UNIX
        shell scripting (including POSIX.2), GNU make, AppleScript, SQL,
        MySQL, \Matlab, Maple, Mathematica, and others
\end{innerlist}

\halfblankline

Version Control and Software Configuration Management:
%
\begin{innerlist}
    \item DVCS (Mercurial/MQ, Git/StGit), VCS (RCS, CVS, SVN, SCCS), and
        others
\end{innerlist}

\halfblankline

\href{http://www.mathworks.com/products/matlab/}{\Matlab} skill set:
%
\begin{innerlist}
    \item Linear algebra, Fourier transforms, Monte Carlo analysis,
        nonlinear numerical methods, polynomials, statistics,
        $N$-dimensional filters, visualization

    \item Toolboxes: communications, control system, filter design,
        genetic algorithm and direct search, signal processing, system
        identification
\end{innerlist}

\halfblankline

Software Verification:
%
\begin{innerlist}
    \item KeY, PRISM, KeYmaera
\end{innerlist}

\halfblankline

Information/Internet Technology:
%
\begin{innerlist}
    \item Networking (UDP, TCP, ARP, DNS, Dynamic
        routing), Services (Apache, SQL, MediaWiki, POP, IMAP, SMTP,
        application-specific daemon design)
\end{innerlist}

\halfblankline

Productivity Applications:
%
\begin{innerlist}
    \item \TeX{} (\LaTeX{}, \BibTeX{}, PSTricks), Vim,
        most common productivity packages (for Windows, OS X, and Linux
        platforms)
\end{innerlist}

\halfblankline

Operating Systems:
%
\begin{innerlist}
    \item Microsoft Windows family, Apple OS X, IBM OS/2, Linux, BSD,
        IRIX, AIX, Solaris, and other UNIX variants
\end{innerlist}

\section{Expertise}
%
Mathematics:
%
\begin{innerlist}
    \item Applied Mathematics, Real and Complex Analysis, Measure
        Theory, Differential Geometry, Game Theory, Graph Theory,
        Combinatorics
\end{innerlist}

\halfblankline

Control Theory and Engineering:
%
\begin{innerlist}
    \item Linear and Nonlinear Systems Theory, Feedback, Variable
        Structure Systems and Sliding Modes, Distributed and Intelligent
        Control, Dynamic Optimization, Bio-mimicry, Bio-inspiration,
        Hybrid and CyberPhysical Systems
\end{innerlist}

\halfblankline

Communications and Signal Processing:
%
\begin{innerlist}
    \item Probability, Random Variables, Stochastic Processes,
        Information Theory, Estimation, Networks
\end{innerlist}

\halfblankline

Computer Science and Engineering:
%
\begin{innerlist}
    \item Model Checking (automated, distributed, hybrid,
        probabilistic), Hybrid Automata, Software Verification,
        Component-Based Reusable Software
\end{innerlist}

\halfblankline

Natural Sciences (Biology, Neuroscience, Psychology, Anthropology):
%
\begin{innerlist}
    \item Behavioral Ecology, Foraging Theory, Cooperation/Altruism,
        Impulsiveness, Evolution
\end{innerlist}

\section{References Available to Contact}%
    Available upon request. Otherwise, see:\\
    \url{http://www.tedpavlic.com/engjobsearch/docs/pavlic\_reference\_list.pdf}

\end{document}

%%%%%%%%%%%%%%%%%%%%%%%%%% End CV Document %%%%%%%%%%%%%%%%%%%%%%%%%%%%%

%----------------------------------------------------------------------%
% The following is copyright and licensing information for
% redistribution of this LaTeX source code; it also includes a liability
% statement. If this source code is not being redistributed to others,
% it may be omitted. It has no effect on the function of the above code.
%----------------------------------------------------------------------%
% Copyright (c) 2007, 2008, 2009, 2010, 2011 by Theodore P. Pavlic
%
% Unless otherwise expressly stated, this work is licensed under the
% Creative Commons Attribution-Noncommercial 3.0 United States License. To
% view a copy of this license, visit
% http://creativecommons.org/licenses/by-nc/3.0/us/ or send a letter to
% Creative Commons, 171 Second Street, Suite 300, San Francisco,
% California, 94105, USA.
%
% THE SOFTWARE IS PROVIDED "AS IS", WITHOUT WARRANTY OF ANY KIND, EXPRESS
% OR IMPLIED, INCLUDING BUT NOT LIMITED TO THE WARRANTIES OF
% MERCHANTABILITY, FITNESS FOR A PARTICULAR PURPOSE AND NONINFRINGEMENT.
% IN NO EVENT SHALL THE AUTHORS OR COPYRIGHT HOLDERS BE LIABLE FOR ANY
% CLAIM, DAMAGES OR OTHER LIABILITY, WHETHER IN AN ACTION OF CONTRACT,
% TORT OR OTHERWISE, ARISING FROM, OUT OF OR IN CONNECTION WITH THE
% SOFTWARE OR THE USE OR OTHER DEALINGS IN THE SOFTWARE.
%----------------------------------------------------------------------%
